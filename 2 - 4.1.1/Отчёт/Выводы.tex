\section*{Обсуждение результатов и выводы}

В работе были определены фокусные расстояния тонких линз:

Положительная линза №1: \\
Метод Аббе: \\
$f = 10,5 \pm 0,3$\\
$D = 9,5 \pm 0,2 \dptr$. \\
Метод Бесселя:\\
$f = 10,25 \pm 0,03 \cm$. \\
$D = 9,80 \pm 0,03 \dptr$. \\
Измерение с помощью зрительной трубы: 
$f = 9,3 \pm 0,1 \cm$. \\
$D = 10,8 \pm 0,1 \dptr$.

Отрицательная линза: \\
Метод Бесселя: \\
$f = -13,1 \pm 0,5 \cm$ \\
$D = -7,6 \pm 0,3 \dptr$. \\
Измерение с помощью зрительной трубы: \\
$f = -14,9 \pm 0,1 \cm$. \\
$D = -6,7 \pm 0,1 \dptr$.

Для положительной линзы №1 результаты определения фокусного расстояния с помощью методов Аббе и Бесселя сходятся в пределах погрешности.

Результаты измерений с помощью зрительной трубы сильно отличаются от измерений с помощью методов Аббе и Бесселя, скорее всего потому, что в при измерения с помощью зрительной трубы нужно было определить такую конфигурацию оптической системы, при которой наблюдается наиболее чёткое изображение. Но точно определить наиболее чёткое изображение не удаётся из-за субъективности человеческого восприятия наблюдаемой картины. Второй причиной расхождения результатов могла быть неточная настройка оптической системы. Луч мог немного отклоняться от главной оптической оси, что приводило к искажению результатов измерений.