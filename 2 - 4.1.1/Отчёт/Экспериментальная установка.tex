\section*{Описание экспериментальной установки}

Определение фокусных расстояний собирающих линз методом Аббе:

\begin{figure}[H]
	\centering
	\includegraphics[width=0.9\textwidth]{../Изображения/Схема установки. Аббе.png}
	\caption{Схема экспериментальной установки}
\end{figure}

Рассматривается несколько различных положений источника на оптической оси статичной линзы. Из геометрической оптики следует связь между его сдвигом и изменением продольного увеличения изображения: \\
f = \frac{\Delta x}{\Delta (y/y')}\\

Определение фокусных расстояний собирающих линз методом Бесселя:

\begin{figure}[H]
	\centering
	\includegraphics[width=0.9\textwidth]{../Изображения/Схема установки. Бессель.png}
	\caption{Схема экспериментальной установки}
\end{figure}

В методе Бесселя при постоянном расстоянии от источника до экрана находятся два положения линзы, при которых получается чёткое изображение. В приближении \delta << L следует формула: \\
f = \frac{L^{2} - l^{2}}{4L}\\

Ислледование системы из двух собирающих линз:

\begin{figure}[H]
	\centering
	\includegraphics[width=0.9\textwidth]{../Изображения/Схема установки. Система собирающих.png}
	\caption{Схема экспериментальной установки}
\end{figure}


Определение фокусного расстояния рассеивающей линзы в сложной системе:

\begin{figure}[H]
	\centering
	\includegraphics[width=0.9\textwidth]{../Изображения/Схема установки. Фокус рассеивающей.png}
	\caption{Схема экспериментальной установки}
\end{figure}

Фокусное расстояние рассеивающей линзы определяется двумя способами:\\
\mathbf{Способ 1:} С помощью собирающей линзы получают действительное изображение источника. Затем между линзой и изображением ставится рассеивающая линза, таким образом для неё изображение становится мнимым источником и изображение от системы остается действительным. Последнее находится с помощью экрана и из расстояний в системе определяется фокусное расстояние.\\
\mathbf{Способ 2:} Система выглядит так же, только рассеивающая линза располагается так, что изображение от собирающей попадает в её фокальную плоскость. В такой конфигурации лучи после неё уходят параллельным пучком, и чтобы найти такое положение, в конец скамьи ставится зрительная труба, настроенная на бесконечность. Тогда фокусное расстояние линзы есть ни что иное, как расстояние между экраном, на котором фокусируется изображение от одной собирающей линзы и полученным положением рассеивающей.\\


\section*{Описание экспериментальной установки}
