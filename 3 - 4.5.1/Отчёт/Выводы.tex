\section*{Обсуждение результатов и выводы}

В данной лабораторной работе мы измерили коэффициент усиления активной среды лазера: результат превысил ожидаемый на 57\%, получили генерацию лазерного излучения и исследовали влияние поляризатора на интенсивность пучка. В качестве теоретической кривой взят закон Малюса: $I = I_0 \cdot  sin^{2}(\phi)$, где $I_0$ - интенсивность при отсутствии поляризатора. Видно, что экспериментальные точки не проходят через ноль, значит, фотодетектор улавливает и другой свет. На втором графике его интенсивность предполагается постоянной и вычтена из всех точек, и, как следствие, на нём уже довольно близки экспериментальная и теоретическая зависимости.Так же изучили модовую структуру лазера.