\section*{Описание экспериментальной установки}

Схема экспериментальной установки для исследования дифракции 
Френеля на щели приведена на рисунке:

\begin{figure}[H]
	\centering
	\includegraphics[width=0.9\textwidth]{../Изображения/Схема
	 установки. Дифракция Френеля.png}
	\caption{Схема экспериментальной установки}
\end{figure}

Пучок света от ртутной лампы(Л) проходит через 
монохроматическую пластинку, из него выделяется световая 
линия средней длины $\lambda = 578 нм$. Дальше полученный 
пучок падает на щель(S$_1$), которая находится в фокусе 
собирающей линзы(O$_1$). Фронт пучка, в данный момент 
проходящий через щель в соответствии с принципом 
Гюйгенса-Френеля рассматривается как источник. Параллельный 
после линзы пучок падает на вторую щель(S$_2$) и испытывает 
на ней дифракцию, которую мы изучаем с помощью микроскопа(М), 
сфокусированного на некоторую плоскость наблюдения(П).\\

Схема экспериментальной установки для исследования дифракции 
Фраунгофера на щели приведена на рисунке:

\begin{figure}[H]
	\centering
	\includegraphics[width=0.9\textwidth]{../Изображения/Схема
	 установки. Дифракция Фраунгофера.png}
	\caption{Схема экспериментальной установки}
\end{figure}

Отличие от предыдущей установки заключается в наличии второй 
линзы(O$_2$), создающей в фокальной плоскости изображение, 
соответствующее бесконечно удалённой плоскости наблюдения. 
Без линзы наблюдаемую плоскость пришлось бы удалить на 
бесконечно большое расстояние от щели, что в лабораторных 
условиях привело бы к крайне низкой освещенности изображения.