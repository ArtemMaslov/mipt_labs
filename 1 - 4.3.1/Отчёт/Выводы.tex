\section*{Обсуждение результатов и выводы}

В работе исследовалась дифракция Френеля и Фраунгофера на 
длинной щели.

При исследовании дифракции Френеля было установлено, что при 
неизменном расстоянии до микроскопа, при увеличении размера 
щели, количество дифракционных минимумов увеличивается, 
они становятся более размытыми и меньшими по толщине.

Наблюдать четкую дифракционную картину на краю экрана не 
удалось. Из-за большого размера входной щели количество 
дифракционных полос на фоне экрана было большим, они были 
размыты. Настроить установку для наблюдения более 
качественной картины не получилось, потому что при 
уменьшении размера входной щели, уменьшалась яркость картинки 
и дифракционные полосы были слабо различимы.

В работе была измерена ширина щели при исследовании дифракции 
Френеля. Слабое расхождение результатов говорит о хорошей 
точности измерений.

\begin{tabular}{|c|c|}
\hline
Микроскоп & 0.300 мм \\
\hline
Микрометрический винт & 0.300 мм \\
\hline
Дифракция Френеля & $0.293 \pm 
0.004$ мм \\
\hline
\end{tabular}

В ходе наблюдений дифракции Фраунгофера было установлено, что 
смещение щели $S_2$ в боковом направлении не приводит к 
сдвигу дифракционной картины.

При исследовании дифракции Фраунгофера были получены 
следующие значения ширины щели:

\begin{tabular}{|c|c|}
	\hline
	Микроскоп & 0.30 мм \\
	\hline
	Микрометр & 0.18 мм \\
	\hline
	Дифракция Фраунгофера & 0.41 $\pm$ 0.06 мм \\
	\hline
\end{tabular}

Результаты сильно расходятся, скорее всего, из-за ошибки, 
допущенной в ходе выполнения работы. Установить точную 
причину расхождения результатов не удалось.