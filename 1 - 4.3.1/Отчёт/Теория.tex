\section*{Теория}

Явление отклонения от распространения света согласно законам геометрической оптики, называется \textit{дифракцией} в широком смысле слове. В узком смысле слова дифракцией называется явление огибания светом препятствия и проникновение в область геометрической тени.

\subsection*{Основная задача дифракции}

При рассмотрении явления дифракции решается следующая основная задача.

\begin{wrapfigure}{left}{0.38\textwidth}
	\centering
	\includegraphics[width=0.38\textwidth]{../Изображения/Задача дифракции.png}
	\caption{Основная задача дифракции}
\end{wrapfigure}

Пусть в плоскости $z = 0$ находится экран с отверстием произвольной формы, и в полупространстве $z < 0$ находятся источники излучения. Пусть известно результирующее поле от источников $f_и(x, y, 0)$, которое они создают в плоскости $z = 0$ в отсутствии экрана. Тогда
\begin{enumerate}
	\item Зная оптические свойства материала, из которого изготовлен экран и геометрические размеры отверстия, необходимо найти распределение поля $f_0(x, y, 0)$ сразу после экрана, в плоскости $z = +0$.
		
	\item По распределению поля на границе экрана $f_0(x, y, 0)$ нужно найти значение поля в области пространства $z > 0$, в частности на исследуемой поверхности $П$, находящейся на расстоянии $z$ от экрана.
\end{enumerate}

\newpage

Получить аналитическое выражение для электромагнитного поля на границе экрана $f_0(x, y, 0)$ часто не возможно. Во-первых, электромагнитная волна на поверхности экрана порождает переменные токи. Электромагнитное поле индуцированных токов необходимо учитывать при вычислении поля на границе экрана. Во-вторых, отверстие может иметь сложную геометрию. В-третьих, оптические свойства материала экрана могут быть не линейными.

Для упрощения решения основной задачи, применяется \textit{приближение Кирхгофа}. Предполагается, что поле в той части экрана, где есть отверстие равно результирующему полю источников $f_и(x, y, 0)$, если бы не было экрана. В области, затененной экраном поле источников считается равным 0. То есть в данном приближении пренебрегается взаимодействием экрана и электромагнитного поля. Данное приближение хорошо описывает реальные физические системы, если, во-первых, линейные размеры отверстия $b$ велики по сравнению с длиной волны $\lambda$: $b > \lambda$. Во-вторых, если плоскость наблюдения находится на расстоянии $z$ много большем длины волны: $z \gg \lambda$.

\subsection*{Принцип Гюйгенса-Френеля}

Согласно \textit{принципу Гюйгенса-Френеля}, каждая точка волнового фронта является источником вторичных сферических волн, а результирующее поле в исследуемой точке пространства --- результат интерференции вторичных волн.

\begin{wrapfigure}{left}{0.4\textwidth}
	\centering
	\includegraphics[width=0.38\textwidth]{../Изображения/Принцип Гюйгенса-Френеля.png}
	\caption{Принцип Гюйгенса-Френеля}
\end{wrapfigure}

Получим качественную оценку для поля в исследуемой точке пространства $P(x, y, z)$. Рассмотрим точку отверстия с координатами $(\xi, \eta)$. Она является источником вторичных сферических волн c амплитудой, пропорциональной амплитуде исходной волны. Фаза вторичной волны равна фазе исходной волны. В точке $P$ амплитуда сферической волны уменьшится в $R$ раз, фаза изменится на $e^{ikR}$. Предполагается, что амплитуда колебаний поля в исследуемой точке пропорциональна видимой площади $ds \cos \alpha$ элементарной площадки, создающей вторичные волны. Итого поле $d g(x, y)$, создаваемое в точке $P$, площадкой $ds$:
$$
dg(x, y) \sim f_0(\xi, \eta) \frac{e^{ikR}}{R} \cos \alpha d\xi d\eta
$$
Интегрируя по всей области отверстия получим выражение для результирующего поля $g(x, y)$ в точке $P$:
\begin{equation*}
	g(x, y) = K_0 \iint\limits_S f_0(\xi, \eta) \frac{e^{ikR}}{R} \cos \alpha \, d\xi \, d\eta
	\label{theory:Huygens–Fresnel-prec}
\end{equation*}
где коэффициент пропорциональности $K_0 = \frac{1}{i \lambda}$. Данное соотношение является количественной формулировкой принципа Гюйгенса-Френеля.

При использовании принципа Гюйгенса-Френеля используют следующие приближения. Во-первых, так как обычно рассматриваются параксиальные лучи, то расстояние $R$ от точек отверстия до всех точек исследуемой плоскости считается одинаковым и равным $R_0$:
$$
g(x, y) = \frac{1}{i \lambda R_0} \iint_S f_0(\xi, \eta) e^{ikR} \cos \alpha  \, d\xi \, d\eta
$$

Для оценки изменения фазы нужно использовать более точную оценку. Разложим расстояние $R$ по формуле Тейлора до второго порядка:
$$
R = \sqrt{z^2 + (x - \xi)^2 + (y - \eta)^2} \approx z + \frac{(x - \xi)^2}{2 z} + \frac{(y - \eta)^2}{2z}
$$
Предполагается, что члены разложения Тейлора более высокого вносят малую поправку в полученный результат. Такое приближение называется \textit{френелевским}. Итого значение поля в точке $P(x, y, z)$ вычисляется по формуле:
\begin{equation*}
	g(x, y) = \frac{e^{ikz}}{i \lambda z} \iint\limits_S f_0(\xi, \eta) e^{i\frac{k}{2z}\left((x-\xi)^2 + (y - \eta)^2\right)} \, d\xi \, d\eta
	\label{theory:Huygens–Fresnel-approx}
\end{equation*}

\subsection*{Дифракция Френеля на круглом отверстии}

\begin{wrapfigure}{left}{0.4\textwidth}
	\centering
	\includegraphics[width=0.38\textwidth]{../Изображения/Дифракция Френеля.png}
	\caption{Дифракция Френеля}
\end{wrapfigure}

Найдем распределение поля в точке $P(0, 0, z)$ от круглого отверстия радиуса $r$. Будем считать, что экран с отверстием освещается параллельным пучком световых волн с одинаковой амплитудой колебаний $A_0$. Тогда преобразуем формулу $\ref{theory:Huygens–Fresnel}$, перейдя к интегрированию по кольцам радиуса $\rho$, толщиной $d\rho$ и площадью $ds = 2 \pi \rho d \rho$:
$$
	g = A_0 \frac{e^{ikz}}{i \lambda z} \int \limits_0^r e^{i\frac{k}{2z} \rho^2} 2\pi \rho d \rho = A_0 e^{ikz} \left( 1 - e^{-ik\frac{k}{2z}r^2} \right)
$$
Интенсивность поля в точке $P$ вычисляется по формуле:
\begin{equation*}
	I \propto <|g|^2> = 2I_0 \left(1 - \cos \left( \frac{k}{2z}r^2 \right)\right)
	\label{eq:Fresnels-intensity}
\end{equation*}
где $I_0$ -- интенсивность исходной волны, усреднение производится за время много больше периода колебаний электромагнитной волны.

Проанализируем полученный результат. Точки $r_m = \sqrt{m \lambda z}, m = 1, 2, 3, \dots$ являются точками экстремума функции интенсивности. При нечётных $m$ наблюдается максимум $I_{max} = 4I_0$, при чётных $m$ -- минимум $I_{min} = 0$.

Полученное соотношение справедливо, когда $\frac{\cos \alpha}{R} \approx \frac{1}{R_0}$. С помощью \textit{метода векторных диаграмм} качественно учтем влияние множителя $\frac{\cos \alpha}{R}$ на результат.

Рассмотрим последовательность колец радиусом $\rho_n$, достаточно малой толщины $d \rho_n << 1$ и одинаковой площади $dS_n = dS_{n+1} = dS$. Тогда вклад от одного кольца в поле в точке $P$ обозначим в виде вектора $d \boldsymbol{A}$ на векторной диаграмме. Модуль этого вектора $dA = A_0 \frac{\cos \alpha}{\lambda R} dS$. Угол наклона вектора относительно горизонтали обозначим за $\phi = \frac{k}{2z} \rho^2$. Угол между двумя соседними векторами $d \phi = \frac{k}{z} \rho d \rho = \frac{1}{\lambda z} dS$ -- постоянная величина. 

\begin{wrapfigure}{left}{0.4\textwidth}
	\centering
	\includegraphics[width=0.38\textwidth]{../Изображения/Векторные диаграммы.png}
	\caption{Метод векторных диаграмм. Спираль Френеля.}
	\label{img:Fresnel-spiral}
\end{wrapfigure}

Векторы $d \boldsymbol{A}$, построенные последовательно друг за другом будут образовывать спираль, медленно скручивающуюся к центру (модуль вектора $dA$ убывает с увеличением радиуса кольца). Результирующее поле $\boldsymbol{A}(\rho)$ в точке $P$ равно сумме вкладов отдельных колец $d\boldsymbol{A}$ (рис. $\ref{img:Fresnel-spiral}$в).

Кольцо $r_{m - 1} < \rho < r_{m}$, где $r_m = \sqrt{m \lambda z}$ называется $m$-ой зоной Френеля. Не трудно показать, что результирующие колебания, создаваемые двумя последовательными зонами Френеля, сдвинуты по фазе на $\pi$. На векторной диаграмме зонам Френеля соответствуют полуокружности. На рисунке $\ref{img:Fresnel-spiral}$а изображена первая зона Френеля, результирующий вектор $A_1 = 2 A_0$. На рисунке $\ref{img:Fresnel-spiral}$б открыты две зоны Френеля и результирующее поле в точке $P$ мало, но не равно нулю.

Если открыто нечётное число зон Френеля, то наблюдается максимум амплитуды, если открыто чётное число зон Френеля, то -- минимум.

\subsection*{Дифракция Френеля на узкой щели}

% Определение.
% Векторные диаграммы + спираль Корню.

\subsection*{Условия, определяющие тип дифракции}

\subsection*{Дифракция Фраунгофера}

% Определение
% Перобразование Фурье.
% Одномерный случай - замена u\xi + v \eta на k \sin \theta \xi
% Обоснование применения линзы.
% Дифракция на щели.
% Дифракция на двух щелях.