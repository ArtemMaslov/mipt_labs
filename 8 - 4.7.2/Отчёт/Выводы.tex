\section*{Обсуждение результатов и выводы}

В работе наблюдалась интерференционная картина обыкновенной и необыкновенных волн, образовавшихся после прохождения кристалла ниобата лития монохроматического поляризованного лазерного излучения.

В работе было измерено двулучепреломление кристалла в отсутствии внешнего электрического поля: \\
$n_o - n_e = 0,097 \pm 0,002$.

Согласно справочнику для длины волны $\lambda = 632,8 \nm$ кристалл ниобата лития имеет показатели преломления $n_o = 2,286$, $n_e = 2,203$. Тогда двулучепреломление: \\
$(n_o - n_e)^{табл} = 0,083$.

В работе было определено полуволновое напряжение образца кристалла ниобата лития и коэффициент пропорциональности изменения показателя преломления пластинки от внешнего электрического поля $A$, $\Delta n = A E_{эл}$:\\
$U_{\frac{\lambda}{2}} = 351 \pm 12$ методом наблюдения периодических изменений интенсивности света на экране при постоянном внешнем электрическом поле внутри пластинки. $A = (52 \pm 2) \cdot 10^{-12} \; \frac{м}{В}$. \\

$U_{\frac{\lambda}{2}} = 390 \pm 8$ методом наблюдения фигур Лиссажу при переменном электрическом поле внутри пластинки. $A = (47 \pm 1) \cdot 10^{-12} \; \frac{м}{В}$.