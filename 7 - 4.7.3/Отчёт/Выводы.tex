\section*{Обсуждение результатов и выводы}

С помощью метода чёрного зеркала было определено направление первого поляроида $\varphi_1 = 87^{\circ} \pm 1^{\circ}$, когда он пропускает горизонтальную волну.

С помощью первого поляроида было определено разрешённое направление второго поляроида $\varphi_2 = 115^\circ \pm 1^{\circ}$, когда он пропускает вертикальную волну.

Был определён показатель преломления эбонита: \\
$n = 1,28 \pm 0,13$. \\
$n_{табл} = 1,6 \div 1,7$.

Расхождение результатов связано с неточностью определения минимальной интенсивности глазом экспериментатора.

В работе были исследованы поляризационные свойства стопы стеклянных пластин. \\
Для света, прошедшего стопу стеклянных пластин, интенсивность горизонтальной компоненты больше интенсивности вертикальной. \\
У света отражённого от стопы стеклянных пластин горизонтальная компонента практически отсутствует, а вертикальная компонента хорошо видна. \\
Интенсивность вертикальной компоненты отражённого от стопы стеклянных пластин света больше интенсивности вертикальной компоненты прошедшего света.

Были определены разрешённые направления двоякопреломляющих пластин: \\
Пластина <<2 в кружочке>>. Направления наименьшей интенсивности: \\
$\varphi_{11} = 16^\circ$. \\
$\varphi_{12} = 110^\circ$. \\
$\varphi_{13} = 196^\circ$. \\
$\varphi_{14} = 288^\circ$. \\
Пластина <<2>>. \\
$\varphi_{11} = 40^\circ$. \\
$\varphi_{12} = 130^\circ$. \\
$\varphi_{13} = 220^\circ$. \\
$\varphi_{14} = 310^\circ$.

Было установлено, что пластина <<2 в кружочке>> -- пластина длины $\frac{\lambda}{4}$, а пластина <<2>> -- пластина $\frac{\lambda}{2}$.

Для пластины $\frac{\lambda}{4}$ было определено, что быстрой оси соответствует угол $\varphi = 155^\circ$ и направление эллиптической поляризации -- правое.