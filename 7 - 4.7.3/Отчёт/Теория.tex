\section*{Теория}

В работе изучаются свойства поляризованного света. В линейно поляризованной световой волне пара векторов $\vec{E}$ и $\vec{H}$ не изменяет с течением времени своей ориентации. Плоскость $\vec{E}$, $\vec{S}$ называется в этом случае плоскостью колебаний. Наиболее общим типом поляризации является \textit{эллиптическая поляризация}. В эллиптически поляризованной световой волне конец вектора $\vec{E}$ (в данной точке пространства) описывает некоторый эллипс.

При теоретическом рассмотрении различных типов поляризации часто бывает удобно проектировать вектор $\vec{E}$ в некоторой точке пространства на два взаимно перпендикулярных направления. В том случае, когда исходная волна была поляризованной, $E_x$ и $E_y$ когерентны между собой и могут быть записаны в виде
\begin{equation}
	\begin{cases}
		E_x = E_{x_0}\cos(kz - \omega t),\\
		E_y = E_{y_0}\cos(kz - \omega t-\varphi),
	\end{cases}
\end{equation}
где амплитуды $E_{x_0}$, $E_{y_0}$, волновой вектор $k$, частота $\omega$ и сдвиг фаз $\varphi$ не зависят от времени. Формулы (1) описывают монохроматический свет. Немонохроматический свет может быть представлен суммой выражений типа (1) с различными значениями частоты $\omega$.

Ориентация эллипса поляризации определяется отношением амплитуд $E_{y_0}/E_{x_0}$ и разностью фаз $\varphi$. В частности, при $\varphi = 0, \pm\pi$ эллипс вырождается в отрезок прямой (линейная поляризация). При $\varphi = \pm\pi/2$ главные оси эллипса совпадают с осями $x$, $y$. Если при этом отношение амплитуд $E_{y_0}/E_{x_0} = 1$, эллипс поляризации вырождается в окружность.	

В плоскости $z = z_0$ вектор $\vec{E}$ волны (1) вращается против часовой стрелки (при наблюдении навстречу волне), если $0 < \varphi < \pi$. В этом случае говорят о левой эллиптической поляризации волны. Если же
$\pi < \varphi < 2\pi$, вращение вектора $\vec{E}$ происходит по часовой стрелке, и волна имеет правую эллиптическую поляризацию.


В фиксированный момент времени $t = t_0$ концы вектора $\vec{E}$ при различных $z$ лежат на винтовой линии. При этом для левой эллиптической поляризации образуется левый винт, а для правой --- правый винт. \\

\textbf{Методы получения линейно поляризованного света.} Для получения линейно поляризованного света применяются \textit{поляризаторы}. Направление колебаний электрического вектора в волне, прошедшей через поляризатор, называется \textit{разрешенным направлением поляризатора}. Всякий поляризатор может быть использован для исследования поляризованного света, т. е. в качестве анализатора. Интенсивность $I$ линейно поляризованного света после прохождения через анализатор зависит от угла, образованного плоскостью колебаний с разрешенным направлением анализатора:
\begin{equation}
	I = I_0 \cos^2\alpha.
\end{equation}
Соотношение (2) носит название закона Малюса. Опишем способы получения плоскополяризованного света, используемые в работе. \\

\textbf{Отражение света от диэлектрической пластинки}. Отраженный от диэлектрика свет всегда частично поляризован. Степень поляризации света, отраженного от диэлектрической пластинки в воздух, зависит от показателя преломления диэлектрика $n$ и от угла падения $i$. Как следует из формул Френеля, полная поляризация отраженного света достигается при падении под углом Брюстера, который определяется соотношением
\begin{equation}
	\text{tg}i = n.
\end{equation}
В этом случае плоскость колебаний электрического вектора в отраженном свете перпендикулярна плоскости падения. \\

\textbf{Преломление света в стеклянной пластинке}. Поскольку отраженный от
диэлектрической пластинки свет оказывается частично (или даже полностью) поляризованным, проходящий свет также частично поляризуется. Преимущественное направление колебаний электрического вектора
в прошедшем свете совпадает с плоскостью преломления луча. Максимальная поляризация проходящего света достигается при падении под
углом Брюстера. Для увеличения степени поляризации преломлённого
света используют стопу стеклянных пластинок, расположенных под углом Брюстера к падающему свету. \\

\textbf{Преломление света в двоякопреломляющих кристаллах}. Некоторые кристаллы обладают свойством двойного лучепреломления. Это связано с различием поляризуемости молекул в разных направлениях (диэлектрическая проницаемость $\varepsilon$ определяет показатель преломления среды $n$).
Двоякопреломляющий кристалл называют одноосным, если в нём существует одно направление с экстремальным значением $\varepsilon$, а в других (перпендикулярных) направлениях значения $\varepsilon$ одинаковы. Направления вдоль осей эллипсоида называют главными, одно из них --- c экстремальным значением $\varepsilon$ --- оптической осью. Преломляясь в таких кристаллах, световой луч разделяется на два луча со взаимно перпендикулярными плоскостями колебаний. Отклоняя	один из лучей в сторону, можно получить плоскополяризованный свет, --- так устроены поляризационные призмы.


