\section*{Обсуждение результатов и выводы}

В работе была определена оптическая сила голограммы $D = 3,88 \pm 0,03 \; дптр$. \\
Было определено фокусное расстояние голограммы $f = 25,8 \pm 0,2 \cm$. \\
С помощью фокусирующих свойств было определено расстояние от голограммы до точечного источника, использованного при её создании: $a = 38 \pm 5 \mm$.

В работе был измерен угол падения опорной волны, использованной при создании голограммы $\varphi = 47^\circ$.

Было проверено свойство голограммы: при закрытии её части непрозрачным листом бумаги, изображение полностью восстанавливалось по оставшейся открытой части.

Было измерено расстояние от голограммы до предметов, использованных при её создании:
Расстояние до линейки $l_1 = 101 \mm$. \\
Расстояние до гвоздя $l_2 = 151 \mm$.