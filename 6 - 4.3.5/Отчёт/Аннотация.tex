\section*{Аннотация}

В работе определяется расстояние от голограммы до точечного источника, который использовался при её создании, двумя методами:
\begin{enumerate}
	\item По результатам измерения радиусов голографических колец, спроецированных на экран при помощи короткофокусной линзы.
	
	\item По результатам измерения параметров проекционной установки, в которой голограмма используется как короткофокусная линза, а объектом служит предметная шкала.
\end{enumerate}

Исследуются свойства голограммы объёмного предмета - линейки и стержня, расположенного за линейкой:
\begin{enumerate}
	\item Оценивается угол падения опорной волны, использованной при создании голограммы.
	
	\item Проверяется, что изображение предмета восстанавливается по части голограммы.
	
	\item Оценивается расстояние от голограммы до линейки и стержня.
\end{enumerate}