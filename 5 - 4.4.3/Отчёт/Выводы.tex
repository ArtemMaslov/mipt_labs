\section*{Обсуждение результатов и выводы}

В работе с помощью гониометра был измерен преломляющий угол призмы $\Delta \alpha = 63^\circ 8' 35 '' \pm 0^\circ 0' 1''$.

Был определёны основные спектральные параметры исследуемого материла и его тип -- Ф1.
\begin{tabular}{|c|c|c|c|c|}
	\hline
 	& $n_e$ & $n_D$ & $n_F - n_C$ & $\nu_D$ \\
	\hline
	Исследуемый образец & 1,6163 & 1,6122 & 0,015 & 39,9 \\
	\hline
	Ф1 & 1,6169 & 1,6128 & 0,01659 & 36,93\\
	\hline
\end{tabular}

Была определена угловая дисперсия призмы в окрестности жёлтой спектральной линии ртути \\
$D = (1,73 \pm 0,03) \cdot 10^{-4}$. \\
Максимальная разрешающая способность призмы $R_1 = (6,4 \pm 0,1) \cdot 10^3$. \\
Разрешающая способность спектрального прибора, состоящего из гониометра, призмы и глаза экспериментатора, $R_2 = (1,1 \pm 0,1) \cdot 10^3$ меньше $R_1$ в примерно 6 раз, потому что система гониометр-призма изготовлены не идеально точно. Также на разрешающую способность влияет острота зрения экспериментатора.